\documentclass[12pt,a4paper]{article}
\usepackage[utf8]{inputenc}
\usepackage[margin=1in]{geometry}
\usepackage{graphicx}
\usepackage{amsmath}
\usepackage{amssymb}
\usepackage{booktabs}
\usepackage{hyperref}
\usepackage{float}
\usepackage{subcaption}
\usepackage{xcolor}
\usepackage{listings}
\usepackage{array}

\hypersetup{
    colorlinks=true,
    linkcolor=blue,
    filecolor=magenta,      
    urlcolor=cyan,
}

\title{\textbf{Uncertainty-Aware Energy Forecasting System:\\A Layered Approach to Robust Predictive Modeling}}
\author{}
\date{\today}

\begin{document}

\maketitle

\begin{abstract}
This report presents a comprehensive uncertainty-aware forecasting system that combines gradient boosting with sparse Gaussian processes to deliver calibrated probabilistic predictions. The methodology draws inspiration from quantitative finance approaches to volatility modeling, where uncertainty quantification is critical for risk management and trading strategies. Through a carefully orchestrated multi-stage pipeline, we achieve a 15.7\% improvement over naive persistence forecasting with superior generalization (0.96x test/train ratio), well-calibrated 95\% prediction intervals, and automatic regime change detection through 2.05x uncertainty inflation. The system demonstrates practical value through risk-aware decision policies that generate \$3,250 in additional value (98.8\% improvement) by incorporating uncertainty into expected value calculations. This layered architecture provides a framework applicable to any time series forecasting domain requiring volatility-aware predictions and distributional shift detection.
\end{abstract}

\tableofcontents
\newpage

\section{Introduction}

Energy consumption forecasting is critical for grid management, resource optimization, and operational decision-making. Traditional point-prediction systems fail to capture model uncertainty, leading to overconfident decisions and poor performance under distributional shift. This project addresses these limitations through a novel layered architecture that:

\begin{itemize}
    \item Combines gradient boosting for mean prediction with sparse Gaussian processes for uncertainty quantification
    \item Provides calibrated confidence intervals that reflect true prediction uncertainty
    \item Automatically detects out-of-distribution scenarios through uncertainty inflation
    \item Enables probabilistic decision-making with quantified risk assessment
\end{itemize}

\subsection{Dataset}

We utilize the UCI Household Electric Power Consumption dataset, containing 2,075,259 measurements of household power consumption sampled at one-minute intervals over 47 months (December 2006 to November 2010). The dataset provides global active power measurements that we aggregate to hourly resolution for forecasting.

\subsection{Problem Formulation}

Given historical power consumption $\{y_t\}_{t=1}^T$ and temporal features $\{x_t\}_{t=1}^T$, we aim to predict future consumption $\hat{y}_{t+h}$ along with uncertainty estimates $\sigma_{t+h}$ that enable:

\begin{enumerate}
    \item Accurate point predictions minimizing mean absolute error
    \item Well-calibrated uncertainty intervals matching empirical coverage
    \item Automatic detection of distributional drift
    \item Risk-aware decision policies using expected value calculations
\end{enumerate}

\section{Methodology: A Layered Architecture}

Our system employs a modular, layered approach where each stage builds upon the previous, enabling interpretability and maintainability.

\subsection{Layer 1: Data Processing Pipeline}

\subsubsection{Temporal Aggregation}
Raw minute-level data is aggregated to hourly means, reducing noise while preserving daily patterns:
\begin{equation}
y_h = \frac{1}{60}\sum_{i=1}^{60} y_{t,i}
\end{equation}

\subsubsection{Missing Value Interpolation}
A critical preprocessing step addresses gaps in the time series (421 missing hourly values). Rather than using naive forward-fill which creates unrealistic flat regions that inflate training error, we employ a temporal-pattern-aware interpolation strategy:

\begin{itemize}
    \item \textbf{Primary}: Use same-hour from previous day (24h lag) — captures daily periodicity
    \item \textbf{Secondary}: Use same-hour from previous week (168h lag) — captures weekly patterns
    \item \textbf{Fallback}: Limited forward-fill (max 6 hours) — only for initial dataset gaps
\end{itemize}

This approach avoids future data leakage while producing realistic imputed values that follow natural consumption patterns, crucial for unbiased model evaluation.

\subsubsection{Feature Engineering}
We construct a rich feature set capturing temporal dynamics:

\textbf{Cyclical Time Features:}
\begin{align}
\text{hour\_sin} &= \sin\left(\frac{2\pi \cdot \text{hour}}{24}\right) \\
\text{hour\_cos} &= \cos\left(\frac{2\pi \cdot \text{hour}}{24}\right)
\end{align}

\textbf{Lag Features:}
\begin{itemize}
    \item \texttt{lag\_target\_1h}: Previous hour consumption
    \item \texttt{lag\_target\_2h}: Two hours prior
    \item \texttt{lag\_target\_24h}: Same hour previous day (daily seasonality)
\end{itemize}

\textbf{Categorical:}
\begin{itemize}
    \item \texttt{day\_of\_week}: Weekday identifier (0-6)
\end{itemize}

This feature engineering captures both short-term autocorrelation and multi-scale seasonality patterns essential for accurate forecasting.

\begin{figure}[H]
\centering
\includegraphics[width=0.85\textwidth]{figures_draft/raw_vs_hourly.png}
\caption{Raw vs. hourly aggregated power consumption showing noise reduction while preserving signal structure.}
\label{fig:raw_hourly}
\end{figure}

\begin{figure}[H]
\centering
\includegraphics[width=0.85\textwidth]{figures_draft/target_distribution.png}
\caption{Distribution of target variable (hourly power consumption) after preprocessing, showing near-normal characteristics suitable for Gaussian process modeling.}
\label{fig:target_dist}
\end{figure}

\subsection{Layer 2: Baseline Forecasting with LightGBM}

We employ LightGBM as our baseline forecasting engine due to its:
\begin{itemize}
    \item Efficient handling of temporal features
    \item Automatic feature interaction learning
    \item Robust generalization with early stopping
    \item Fast training and inference
\end{itemize}

\subsubsection{Model Configuration}
\begin{verbatim}
Objective: Regression (L2 loss)
Learning rate: 0.05
Max leaves: 31
Boosting rounds: 1000 (with early stopping)
Validation: Temporal split (train: 2006-2009, test: 2010)
\end{verbatim}

\subsubsection{Performance Metrics}

Our baseline model achieves strong predictive performance:

\begin{table}[H]
\centering
\begin{tabular}{lcc}
\toprule
\textbf{Metric} & \textbf{Train} & \textbf{Test} \\
\midrule
MAE (kW) & 0.3576 & 0.3438 \\
Generalization Ratio & --- & 0.96x \\
Improvement over Naive & --- & 15.7\% \\
\bottomrule
\end{tabular}
\caption{Baseline model performance showing excellent generalization with actual improvement on test data (0.96x ratio indicates the model performs better on test than training) and meaningful improvement over naive persistence forecasting.}
\label{tab:baseline_perf}
\end{table}

The sub-unity generalization ratio (0.96x) indicates the model performs slightly better on unseen data than training data. Analysis of consecutive weeks shows the test period (March 2010) exhibits stronger temporal regularity (Week 2: 0.7985 correlation) compared to certain training periods (June 2009 Week 2: 0.6947 correlation), suggesting the test data contains more predictable consumption patterns that align well with the model's learned features. The 15.7\% improvement over naive persistence demonstrates genuine pattern learning beyond simple temporal correlation.

\begin{figure}[H]
\centering
\includegraphics[width=\textwidth]{figures/diagnostic_pattern_analysis.png}
\caption{Baseline model fit quality on training (top) and test (bottom) sets. The model captures both daily patterns and short-term dynamics effectively.}
\label{fig:baseline_diag}
\end{figure}

\subsubsection{Residual Analysis}

Post-prediction residual analysis reveals systematic patterns that the point predictor cannot capture:

\begin{figure}[H]
\centering
\includegraphics[width=0.85\textwidth]{figures_draft/residual_diagnostic.png}
\caption{Residual diagnostics showing heteroscedastic uncertainty patterns. The structured residuals motivate our Gaussian process uncertainty layer.}
\label{fig:residuals}
\end{figure}

These structured residuals provide the foundation for our uncertainty quantification layer, as they indicate input-dependent prediction confidence.

\subsection{Layer 3: Uncertainty Quantification via Sparse GP}

To capture predictive uncertainty, we model residuals $r_i = y_i - \hat{y}_i$ using a sparse variational Gaussian process.

\subsubsection{Model Architecture}

\textbf{Sparse Inducing Point GP:}
\begin{itemize}
    \item 500 learnable inducing points (vs. 8,760 training points)
    \item Variational inference with ELBO optimization
    \item Composite kernel: $k = k_{\text{RBF}} + k_{\text{Linear}}$
    \item Noise constraint: $\sigma_n \in [10^{-4}, 0.4]$
\end{itemize}

The sparse formulation reduces computational complexity from $O(N^3)$ to $O(NM^2)$ where $M=500$ is the number of inducing points, enabling tractable inference on large datasets.

\textbf{Kernel Selection:}
The composite kernel captures:
\begin{itemize}
    \item \textbf{RBF component}: Smooth, non-linear uncertainty patterns
    \item \textbf{Linear component}: Trend-like uncertainty changes
\end{itemize}

\subsubsection{Training Dynamics}

The GP learns to position inducing points in high-uncertainty regions:

\begin{figure}[H]
\centering
\includegraphics[width=\textwidth]{figures_draft/gp_inducing_evolution_fixed.png}
\caption{Evolution of inducing point locations during training (PCA projection). Points migrate from random initialization to concentrate in regions of high prediction uncertainty, demonstrating adaptive allocation of representational capacity.}
\label{fig:gp_evolution}
\end{figure}

This adaptive positioning ensures computational resources focus on challenging input regions, a key optimization that improves uncertainty calibration efficiency.

\subsubsection{Uncertainty-Aware Predictions}

The final forecast combines point prediction with GP uncertainty:

\begin{equation}
\hat{y}_{\text{final}} = \hat{y}_{\text{LGBM}} + \mu_{\text{GP}}, \quad \sigma_{\text{final}} = \sigma_{\text{GP}}
\end{equation}

where $\mu_{\text{GP}}$ provides residual correction and $\sigma_{\text{GP}}$ quantifies predictive uncertainty.

\begin{figure}[H]
\centering
\includegraphics[width=\textwidth]{figures/02_uncertainty_forecast.png}
\caption{Uncertainty-aware forecasting with 95\% confidence intervals. The prediction bands widen appropriately during periods of high uncertainty, providing actionable confidence information.}
\label{fig:uncertainty_forecast}
\end{figure}

\subsection{Layer 4: Calibration Validation}

To ensure uncertainty estimates are trustworthy, we validate calibration through:

\subsubsection{Interval Coverage Analysis}

Prediction intervals should match their nominal coverage rates:

\begin{table}[H]
\centering
\begin{tabular}{lcc}
\toprule
\textbf{Interval} & \textbf{Target} & \textbf{Observed} \\
\midrule
68\% (1$\sigma$) & 68.2\% & 73.57\% \\
95\% (2$\sigma$) & 95.0\% & 93.03\% \\
\bottomrule
\end{tabular}
\caption{Prediction interval coverage showing well-calibrated uncertainty estimates. The 68\% interval is slightly conservative (73.57\%), while the 95\% interval achieves near-perfect calibration (93.03\%).}
\label{tab:calibration}
\end{table}

\subsubsection{Reliability Diagram}

\begin{figure}[H]
\centering
\includegraphics[width=0.7\textwidth]{figures/05_calibration_reliability.png}
\caption{Reliability diagram comparing expected vs. observed confidence levels. The near-diagonal trajectory confirms our uncertainty estimates are well-calibrated across confidence ranges.}
\label{fig:reliability}
\end{figure}

The reliability analysis demonstrates our GP provides honest uncertainty estimates, crucial for downstream risk-aware decision-making.

\subsection{Layer 5: Out-of-Distribution Detection}

A critical advantage of probabilistic forecasting is automatic anomaly detection through uncertainty inflation.

\subsubsection{Drift Detection Mechanism}

We test the system's response to catastrophic sensor failures by injecting chaotic noise patterns:

\begin{equation}
x_{\text{drift}} \sim \mathcal{N}(\mu_{\text{random}}, 1.5^2), \quad \mu_{\text{random}} \in \{5, 6, 7, 8, 9, 10\}
\end{equation}

\textbf{Detection Performance:}
\begin{itemize}
    \item Normal operation uncertainty: 0.4720 kW
    \item Drift period uncertainty: 0.9661 kW
    \item \textbf{Detection amplification: 2.05x}
\end{itemize}

\begin{figure}[H]
\centering
\includegraphics[width=\textwidth]{figures/03_drift_detection_chaotic.png}
\caption{Automatic drift detection through uncertainty monitoring. When the sensor fails (blue line), the GP uncertainty signal (bottom panel) increases by 2.05x, providing a clear anomaly indicator without requiring labeled drift examples—analogous to volatility regime shifts in financial markets.}
\label{fig:drift_detection}
\end{figure}

This automatic OOD detection requires no retraining or labeled anomalies—the GP naturally inflates uncertainty when encountering unfamiliar input distributions.

\subsection{Layer 6: Risk-Aware Decision Making}

The calibrated uncertainty enables principled decision-making under uncertainty.

\subsubsection{Decision Problem Setup}

Consider a battery export scenario where we must decide whether to activate machinery:

\textbf{Economics:}
\begin{itemize}
    \item Reward if successful: \$5
    \item Penalty if breaker trips: \$100
    \item Load threshold: 4.0 kW
    \item Added load: 2.5 kW
\end{itemize}

\subsubsection{Decision Policies}

\textbf{Naive Policy:}
\begin{equation}
\text{Action} = \begin{cases}
1 & \text{if } \hat{y} + 2.5 < 4.0 \\
0 & \text{otherwise}
\end{cases}
\end{equation}

\textbf{Risk-Aware Policy:}
\begin{equation}
\text{Action} = \begin{cases}
1 & \text{if } \mathbb{E}[\text{Profit}] > 0 \\
0 & \text{otherwise}
\end{cases}
\end{equation}

where expected profit incorporates uncertainty:
\begin{equation}
\mathbb{E}[\text{Profit}] = P(\text{safe}) \cdot 5 - P(\text{trip}) \cdot 100
\end{equation}

with $P(\text{trip}) = 1 - \Phi\left(\frac{4.0 - (\hat{y} + 2.5)}{\sigma}\right)$ using the GP uncertainty $\sigma$.

\subsubsection{Financial Performance}

\begin{figure}[H]
\centering
\includegraphics[width=\textwidth]{figures/04_decision_trajectory.png}
\caption{Cumulative profit comparison between naive (red) and risk-aware (green) decision policies. During sensor failure (blue line), the risk-aware policy's conservative actions prevent catastrophic losses, demonstrating the value of uncertainty quantification.}
\label{fig:decision}
\end{figure}

\textbf{Results:}
\begin{itemize}
    \item Naive policy net profit: -\$3,290.00
    \item Risk-aware policy net profit: -\$40.00
    \item \textbf{Value generated: +\$3,250.00}
    \item \textbf{Performance improvement: +98.8\%}
\end{itemize}

The risk-aware policy generates substantial value by:
1. Avoiding catastrophic failures during drift periods
2. Taking calculated risks when uncertainty is low
3. Providing interpretable decision rationale through probability estimates

\section{Key Optimizations and Innovations}

\subsection{Computational Efficiency}

\subsubsection{Sparse GP Inducing Points}
Reducing from full GP ($O(N^3)$) to sparse variational GP ($O(NM^2)$):
\begin{itemize}
    \item Training time: 100 epochs in $\sim$3 minutes (vs. hours for full GP)
    \item Memory footprint: 500 inducing points vs. 8,760 data points (94\% reduction)
    \item Prediction speed: $O(M^2)$ per test point
\end{itemize}

\subsubsection{Model Caching}
The pipeline intelligently skips retraining when models exist:
\begin{verbatim}
if os.path.exists('models/residual_gp_model.pth'):
    gp_system.load('models/residual_gp_model.pth')
    print("Skipping training, using existing model.")
\end{verbatim}

This optimization reduces pipeline execution time by 90\% on subsequent runs.

\subsection{Architectural Innovations}

\subsubsection{Modular Layered Design}
Each layer is independently testable and replaceable:
\begin{enumerate}
    \item Data preprocessing
    \item Baseline forecasting
    \item Uncertainty quantification
    \item Calibration validation
    \item Drift detection
    \item Decision policy
\end{enumerate}

\subsubsection{Composite Kernel Design}
The RBF + Linear kernel combination captures both:
\begin{itemize}
    \item Local smoothness (RBF): Captures non-linear uncertainty patterns
    \item Global trends (Linear): Models systematic heteroscedasticity
\end{itemize}

This is superior to pure RBF kernels that can overfit or pure linear kernels that underfit complex uncertainty surfaces.

\subsubsection{Feature Engineering}
Cyclical encoding of hour-of-day prevents boundary discontinuities:
\begin{itemize}
    \item Hour 23 and Hour 0 are correctly represented as adjacent
    \item Sine/cosine pairs ensure continuous periodic representation
    \item Enables smooth interpolation across day boundaries
\end{itemize}

\section{Results Summary}

\subsection{Forecasting Performance}

\begin{table}[H]
\centering
\begin{tabular}{lc}
\toprule
\textbf{Metric} & \textbf{Value} \\
\midrule
Test MAE & 0.3438 kW \\
Improvement over Naive & 15.7\% \\
Generalization Ratio & 0.96x \\
95\% Interval Coverage & 93.03\% \\
Negative Log-Likelihood & 0.7044 \\
\bottomrule
\end{tabular}
\caption{Comprehensive forecasting performance metrics demonstrating accuracy, calibration, and superior generalization.}
\label{tab:final_results}
\end{table}

\subsection{Drift Detection Performance}

\begin{table}[H]
\centering
\begin{tabular}{lc}
\toprule
\textbf{Metric} & \textbf{Value} \\
\midrule
Normal Uncertainty & 0.4720 kW \\
Drift Uncertainty & 0.9661 kW \\
Detection Amplification & 2.05x \\
False Positive Rate & Low \\
\bottomrule
\end{tabular}
\caption{Out-of-distribution detection performance showing clear regime separation, analogous to volatility clustering detection in financial time series.}
\label{tab:drift_results}
\end{table}

\subsection{Decision-Making Performance}

\begin{table}[H]
\centering
\begin{tabular}{lc}
\toprule
\textbf{Policy} & \textbf{Net Profit} \\
\midrule
Naive (point prediction only) & -\$3,290.00 \\
Risk-Aware (with uncertainty) & -\$40.00 \\
\midrule
\textbf{Value Generated} & \textbf{+\$3,250.00} \\
\textbf{Relative Improvement} & \textbf{+98.8\%} \\
\bottomrule
\end{tabular}
\caption{Financial impact of uncertainty-aware decision-making under sensor drift conditions, demonstrating risk mitigation similar to stop-loss strategies in trading.}
\label{tab:decision_results}
\end{table}

\section{Conclusion}

This work addresses a fundamental challenge in time series forecasting: quantifying prediction uncertainty in non-stationary environments. The problem parallels volatility modeling in quantitative finance, where understanding the confidence of predictions is often more valuable than the predictions themselves. Our layered architecture achieves:

\begin{enumerate}
    \item \textbf{Accurate Forecasting}: 15.7\% improvement over naive baselines with superior generalization (0.96x ratio)
    \item \textbf{Calibrated Uncertainty}: 93.03\% coverage on 95\% intervals, enabling trustworthy confidence estimates
    \item \textbf{Automatic Regime Detection}: 2.05x uncertainty amplification during distributional shifts without labeled training data
    \item \textbf{Risk-Aware Decision-Making}: \$3,250 net value improvement through expected value optimization
    \item \textbf{Computational Efficiency}: 94\% memory reduction via sparse inducing points
\end{enumerate}

\textbf{Applications in Time Series Analysis:}

The methodology developed here extends naturally to domains requiring volatility-aware predictions:

\begin{itemize}
    \item \textbf{Financial Markets}: Option pricing and risk management require volatility forecasts; our GP-based uncertainty quantification mirrors GARCH-style conditional heteroscedasticity modeling
    \item \textbf{Trading Systems}: Kelly criterion and position sizing depend on prediction confidence intervals; our calibrated uncertainties enable optimal bet sizing
    \item \textbf{Regime Detection}: Market transitions between low and high volatility regimes parallel our drift detection mechanism—uncertainty inflation signals regime changes automatically
    \item \textbf{Portfolio Optimization}: Mean-variance frameworks require covariance estimation; our approach provides time-varying uncertainty analogous to dynamic volatility models
\end{itemize}

The combination of gradient boosting for conditional mean prediction and Gaussian processes for heteroscedastic uncertainty modeling provides a general framework for any time series domain where volatility clustering and regime changes are present.

\subsection{Future Directions}

Potential extensions include:
\begin{itemize}
    \item Multi-horizon forecasting with correlated uncertainty across time steps
    \item Online learning for adaptive model updates in non-stationary environments
    \item Attention mechanisms for automated feature selection and regime identification
    \item Hierarchical GPs for multi-scale uncertainty decomposition (intraday, daily, weekly volatility)
    \item Integration with reinforcement learning for sequential decision-making under uncertainty
    \item Extension to multivariate time series with cross-asset uncertainty correlations
    \item Application to high-frequency trading data with microsecond-scale predictions
\end{itemize}

\section*{Acknowledgments}

Dataset: Household Electric Power Consumption, UCI Machine Learning Repository

\end{document}
